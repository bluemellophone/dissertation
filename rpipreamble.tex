\usepackage{graphicx}
\usepackage[usenames,dvipsnames,svgnames,table]{xcolor}
\usepackage{pdfpages}
\usepackage{amsmath}
\usepackage{amsfonts}
\usepackage{amsthm}
\usepackage{amssymb}
\usepackage{cite}
\usepackage{subfig}
\usepackage{mathtools}
\usepackage{algorithm}
\usepackage{url}
\usepackage{verbatim}
\usepackage{enumerate}
\usepackage{tabularx}
\usepackage{libertine}
\usepackage{pgf}
\usepackage{tikz,pgfplots}
\pgfplotsset{compat=1.15}
\usetikzlibrary{arrows}
\usetikzlibrary{shapes}
\usetikzlibrary{positioning}
\usetikzlibrary{backgrounds}
\usepackage{pgfgantt}
\usepackage{scrextend}
\usepackage{epsfig}
\usepackage{silence}
\usepackage{float}
\usepackage{ifxetex,ifluatex}
\usepackage{etoolbox}
\usepackage[utf8]{inputenc}
\usepackage[T1]{fontenc}
\usepackage{lmodern}
\usepackage{microtype}
\usepackage{multirow}
\usepackage{times}
\usepackage{xfrac}
\usepackage[super]{nth}
\usepackage{enumitem}
\usepackage[normalem]{ulem}
\usepackage[english]{babel}
\usepackage{alphalph}
\usepackage{siunitx}
\usepackage{tcolorbox}

%%%%%%%%%%%%%%%%

\newcommand{\bb}{\mathbb}

\newcommand{\DIM}{D }
\newcommand{\see}[1]{see \cref{#1}}
\newcommand{\See}[1]{See \cref{#1}}
\newcommand{\Refin}[1]{Referenced in \cref{#1}}
\newcommand{\refin}[1]{referenced in \cref{#1}}
\newcommand{\Imgcite}[1]{Image from \cite{#1}}
\def\citedash{]--[}
\def\citepunct{], [}

\newcommand{\topic}[1]{\subsubsection{#1}}

\renewcommand{\cal}[1]{\mathcal{#1}}
\newcommand{\superscript}[1]{\ensuremath{^{\textrm{#1}}}}
\newcommand{\subscript}[1]{\ensuremath{_{\textrm{#1}}}}
\newcommand{\ds}{\displaystyle}
\newcommand{\dsi}{\displaystyle \hspace*{5mm}}
\newcommand{\ind}{\ensuremath{\hspace*{5mm}}}
\newcommand{\txt}[1]{\textrm{#1}}
\newcommand{\txtbf}[1]{\textbf{#1}}
\newcommand{\ttxt}[1]{{\tt #1}}
\renewcommand{\th}{\superscript{th} }
\newcommand{\nd}{\superscript{nd} }
%\newcommand{\st}{\superscript{st} }
\newcommand{\bincase}[1]{
    \ensuremath{
        \begin{cases}
            1 & \txt{if } #1    \\
            0 & \txt{otherwise}
        \end{cases}
    }}
\newcommand{\bincases}[2]{
    \ensuremath{
        \begin{cases}
            #1 & \txt{if } #2    \\
            0  & \txt{otherwise}
        \end{cases}
    }}

\newcommand{\tight}[1]{\mkern-6mu#1\mkern-6mu }
\newcommand{\ttimes}{\tight{\times}}
\newcommand{\teq}{\tight{=}}
\newcommand{\teqq}{\tight{==}}

%%%%%%%%%%%%%%%%
% Constants
\newcommand{\PTIME}{{\tt PTIME}}


%%%%%%%%%%%%%%%%
% MATH MACROS
\newcommand{\paren}[1]{\left( #1 \right)}
\newcommand{\curly}[1]{\left\{ #1 \right\}}
\newcommand{\func}[2]{\mathbin{#1}\paren{#2}}

%%%%%%%%%%%%%%%%
% Operators
% 
% Logic
\newcommand{\assign}{:=}
\newcommand{\eq}{=}
%\newcommand{\where}{\;\big\vert\;}
\newcommand{\where}{\;\colon\;}
\newcommand{\given}{\;\big\vert\;}
\newcommand*\AND{\txtbf{ and } }
\newcommand*\OR{\txtbf{ or } }
\newcommand*\xor{\mathbin{\oplus}}
\newcommand{\union}{\cup}
\newcommand{\Union}{\bigcup}
\newcommand{\intersect}{\cap}
\renewcommand{\mod}{\txt{mod}}
\newcommand{\Normal}{\cal{N}}
\newcommand{\eye}{\mat{I}}

% Probability Operator
%\newcommand{\probop}{\mathbin{\operatorname{\mathbb{P}}}}
%\newcommand{\probop}{\mathbin{\operatorname{Pr}}}
\newcommand{\probop}{\mathbin{\operatorname{p}}}
\newcommand{\expectop}{\mathbb{E}}
\newcommand{\parzenop}{\mathbin{\hat{\probop}}}
\newcommand{\Parz}[1]{\parzenop\paren{#1}}
\newcommand{\logop}{\mathbin{\operatorname{log}}}
\newcommand{\lnop}{\mathbin{\operatorname{ln}}}
\newcommand{\expop}{\mathbin{\operatorname{exp}}}
\newcommand{\distri}{\sim}

% Convolution
% \newcommand{\conv}{\mathop{\scalebox{1.5}{\raisebox{-0.2ex}{$\ast$}}}}
\newcommand{\laplace}{\nabla^2}
\newcommand{\del}{\Delta}
% Probability
%\renewcommand{\Pr}[1]{\ensuremath{\txt{Pr} \left(#1\right)}}
%\renewcommand{\Pr}{\mathbb{P}}
\renewcommand{\Pr}[1]{\func{\probop}{#1}}
\newcommand{\Prs}[2]{\func{\probop_{#1}}{#2}}
\newcommand{\Ex}[1]{\ensuremath{\expectop \left[#1\right]}}
\newcommand{\ExUnder}[2]{\ensuremath{\underset{#1}{\expectop \left[#2\right]}}}
\newcommand{\ExSub}[2]{\ensuremath{\expectop_{#1} \left[#2\right]}}
% Optimization
\newcommand{\argmax}[1]{\underset{#1}{\operatorname{argmax}}\;}
% \newcommand{\argmin}[1]{\underset{#1}{\operatorname{argmin}}\;}
\newcommand{\localmax}[1]{\underset{#1}{\operatorname{localmax}}}
\newcommand{\arglocalmax}[1]{\underset{#1}{\operatorname{localmax}}}
\newcommand{\arglocalextrema}[1]{\underset{#1}{\operatorname{extrema}}}
% Calculation
\renewcommand{\exp}[1]{\txt{exp}\left(#1\right)}
\renewcommand{\ln}[1]{\txt{ln}\left(#1\right)}

\newcommand{\overbar}[1]{\mkern 1.5mu\overline{\mkern-1.5mu#1\mkern-1.5mu}\mkern 1.5mu}

% Linear Algebra
%\newcommand{\sqrtm}[1]{\operatorname{sqrtm}(#1)}
\newcommand{\sqrtm}[1]{#1^{\frac{1}{2}}}
\newcommand{\tr}{\txt{Tr}}
\renewcommand{\det}{\txt{Det}}
\newcommand{\cov}{\Sigma}
\newcommand{\ltwonormvec}[1]{\frac{#1}{\elltwo{#1}}}
\newcommand{\card}[1]{|#1|}
\newcommand{\braket}[2]{\left<#1|#2\right>}
\newcommand{\bra}[1]{\left<#1|}
\newcommand{\ket}[2]{|#1\right>}
%\newcommand{\card}[1]{\txt{card}(#1)} 
%\newcommand{\sqrtm}[1]{#1^(.5)}


%%%%%%%%%%%%%%%%
% Variables
% 

% Better ensure math with a space  Actually no. just use \cmd\
%\newcommand{\enmath}[1]{\ensuremath{#1}\xspace}

\newcommand{\mat}[1]{\ensuremath{\mathbf{#1}}} %Should always be capital
\newcommand{\rand}[1]{\ensuremath{\mathbf{#1}}}  %Should always be capital
\newcommand{\set}[1]{\ensuremath{\mathcal{#1}}}  %Should always be lowercase
\let\arrowvec\vec % Keep old arrowvec functionality
\renewcommand{\vec}[1]{\ensuremath{\mathbf{#1}}} %Should always be lowercase
% An explicit vector 
%  e.g. \VEC{ x \\ y \\ z}
\newcommand{\VEC}[1]{\ensuremath{
        \Bigl[\negthinspace \begin{smallmatrix} #1
            \end{smallmatrix} \negthinspace \Bigr] }}
\newcommand{\BVEC}[1]{
    \begin{bmatrix}
        #1
    \end{bmatrix}}
% An explicit matrix 
% e.g. \MAT{ a & b \\ c & d}
\newcommand{\MAT}[1]{\ensuremath{
        \Bigl[ \begin{smallmatrix} #1
            \end{smallmatrix} \Bigr] }}


\newcommand{\REAL}{\mathbb{R}}
%\renewcommand{\Pr}{\color{red} \txtbf{{Pr}} \color{black}}

\newcommand{\elltwo}[1]{||#1||_2}
\newcommand{\ltwo}[1]{||#1||_2}
\newcommand{\lone}[1]{||#1||_1}
\newcommand{\len}[1]{|#1|}
\newcommand{\abs}[1]{|#1|}
%\newcommand{\tau}[1]{|#1|}


%%%%%%%%%%%%%%%%
% Common notation
\newcommand{\TP}{\tt{TP}}
\newcommand{\TN}{\tt{TN}}
\newcommand{\FP}{\tt{FP}}
\newcommand{\FN}{\tt{TN}}

\newcommand{\boldgreek}[1]{\mbox{\boldmath $ #1 $}}

\newcommand{\eps}{\epsilon}
\newcommand{\prefers}{\succ}
\newcommand{\preferseq}{\succeq}


%%%%%%%%%%%%%%%%
% Gantt Charts

% select a colour for the shading
\definecolor{grey}{HTML}{F0F0F0}
\colorlet{shadecolor}{grey}

\newganttlinktype{rdldr*}{%
    \draw [/pgfgantt/link]
    (\xLeft, \yUpper) --
    (\xLeft + \ganttvalueof{link bulge 1} * \ganttvalueof{x unit},
    \yUpper) --
    ($(\xLeft + \ganttvalueof{link bulge 1} * \ganttvalueof{x unit},
    \yUpper)!%
    \ganttvalueof{link mid}!%
    (\xLeft + \ganttvalueof{link bulge 1} * \ganttvalueof{x unit},
    \yLower)$) --
    ($(\xRight - \ganttvalueof{link bulge 2} * \ganttvalueof{x unit},
    \yUpper)!%
    \ganttvalueof{link mid}!%
    (\xRight - \ganttvalueof{link bulge 2} * \ganttvalueof{x unit},
    \yLower)$) --
    (\xRight - \ganttvalueof{link bulge 2} * \ganttvalueof{x unit},
    \yLower) --
    (\xRight, \yLower);%
}

\ganttset{
    link bulge 1/.link=/pgfgantt/link bulge,
    link bulge 2/.link=/pgfgantt/link bulge
}

\renewcommand*{\thesubfigure}{ \alphalph{\value{subfigure}} }

\def\httilde{\mbox{\tt\raisebox{-.5ex}{\symbol{126}}}}

\renewcommand{\tabularxcolumn}[1]{>{\small}m{#1}}
\DeclareMathOperator*{\argmin}{arg\,min}
\newcommand{\vect}[1]{{\bf #1}}                 %for bold chars
\newcommand{\vecg}[1]{\mbox{\boldmath $ #1 $}}  %for bold greek chars
\newcommand{\matx}[1]{{\bf #1}}
\newcommand{\mynew}{}           % for handouts
\newcommand{\specialcell}[2][c]{\begin{tabular}[#1]{@{}c@{}}#2\end{tabular}}

% select a colour for the shading
\definecolor{grey}{HTML}{F0F0F0}
\definecolor{darkgreen}{HTML}{398746}
\colorlet{shadecolor}{grey}

\tikzset{%
    every neuron/.style={
            circle,
            draw,
            minimum size=1cm
        },
    neuron missing/.style={
            draw=none,
            scale=4,
            text height=0.333cm,
            execute at begin node=\color{black}$\vdots$
        },
}

\DeclarePairedDelimiter\ceil{\lceil}{\rceil}
\DeclarePairedDelimiter\floor{\lfloor}{\rfloor}

\newcommand*{\head}[1]{%
    \textbf{#1}
}

\renewcommand{\textfraction}{0.01}
\renewcommand{\floatpagefraction}{0.99}
\renewcommand{\topfraction}{0.99}
\renewcommand{\bottomfraction}{0.99}
\renewcommand{\dblfloatpagefraction}{0.99}
\renewcommand{\dbltopfraction}{0.99}
\renewcommand{\th}{$^{th}$}
\newcommand{\conv}{\mathop{\scalebox{1.2}{\raisebox{-0.2ex}{$\ast_{x,y}$}}}}

\newlength{\subfigheight} % REMOVE?
\setlength{\subfigheight}{1in} % REMOVE?

\newcommand{\cnote}[1]{%
    {\color{red}[#1]}
}

%A bunch of definitions that make my life easier
\newcommand{\R}{\mathbb{R}}
\newcommand{\Z}{\mathbb{Z}}

%%%%%%%%%%%%%%%%
% Lemmas

\theoremstyle{definition}
\newtheorem{theorem}{Theorem}[section]
\newtheorem{lemma}{Lemma}[section]

% \numberwithin{equation}{section}

\newcommand{\eproof}{\hfill\qedsymbol}
\newcommand\ddfrac[2]{\frac{\displaystyle #1}{\displaystyle #2}}

%%%%%%%%%%%%%%%%
% Squished Lists

\newcounter{Lcount}
\newcommand{\numsquishlist}{
    \begin{list}{\arabic{Lcount}. }
        { \usecounter{Lcount}
            \setlength{\itemsep}{-.1ex}      \setlength{\parsep}{0ex}
            \setlength{\topsep}{0.5ex}       \setlength{\partopsep}{0.5ex}
            \setlength{\leftmargin}{2em} \setlength{\labelwidth}{1em}
            \setlength{\labelsep}{0.1em} } }
        \newcommand{\numsquishend}{\end{list}}

\newcommand{\squishlist}{
    \begin{list}{$\bullet$}
        { \setlength{\itemsep}{-.1ex}      \setlength{\parsep}{0ex}
            \setlength{\topsep}{0ex}       \setlength{\partopsep}{0ex}
            \setlength{\leftmargin}{2em} \setlength{\labelwidth}{1em}
            \setlength{\labelsep}{0.5em} } }
        \newcommand{\squishend}{\end{list}}

\newcommand\blfootnote[1]{%
    \begingroup
    \renewcommand\thefootnote{}\footnote{#1}%
    \addtocounter{footnote}{-1}%
    \endgroup
}
